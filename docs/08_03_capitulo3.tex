%%%%%%%%%%%%%%%%%%%%%%%%%%%%%%%%%%%%%%%%%%%%%%%%%%%%%%%%%%%%%%%%%%%%%%%%%%%%%%%

\chapter{Conclusão}

O projeto CPTECRAFT evidenciou ser uma iniciativa altamente inovadora para a popularização da ciência e o estímulo ao interesse científico entre crianças e adolescentes, combinando tecnologias acessíveis com metodologias lúdicas e gamificadas. A réplica virtual detalhada do CPTEC/INPE foi um ambiente eficaz para aprendizagem ativa e imersiva, suportada pelo uso da realidade virtual com óculos confeccionados de forma sustentável~\citeonline{araujo2016}.

Os resultados obtidos, tanto qualitativos quanto quantitativos, confirmam a aceitação e o potencial transformador da proposta. A identificação de desafios técnicos e pedagógicos aponta para oportunidades importantes de aperfeiçoamento, essenciais para ampliar o alcance e o impacto do projeto.

Com ações previstas de expansão para múltiplas plataformas, melhor suporte técnico e formação docente, o CPTECRAFT se posiciona como um modelo exemplar de inovação educacional, capaz de transformar a forma como a ciência é vivenciada e aprendida pelas novas gerações, fomentando uma sociedade mais informada e tecnologicamente preparada para os desafios do futuro.
\newpage