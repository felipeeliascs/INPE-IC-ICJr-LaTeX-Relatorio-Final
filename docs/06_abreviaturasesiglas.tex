%%%%%%%%%%%%%%%%%%%%%%%%%%%%%%%%%%%%%%%%%%%%%%%%%%%%%%%%%%%%%%%%%%%%%%
%% Sugestão de prompt para IA:
%% "Crie um bloco LaTeX para a seção 'Abreviaturas e Siglas' no ambiente \begin{abreviaturasesiglas} ... \end{abreviaturasesiglas}, 
%% formatando siglas e seus significados com separador '&--&' e quebra de linha '\\'. A lista deve conter abreviações comuns e seus significados, 
%% podendo incluir termos em português e explicações em inglês entre parênteses, respeitando sintaxe LaTeX. {COLE A LISTA DE SIGLAS AQUI}"
%%%%%%%%%%%%%%%%%%%%%%%%%%%%%%%%%%%%%%%%%%%%%%%%%%%%%%%%%%%%%%%%%%%%%%
%%%%Exemplo:
%%%%\begin{abreviaturasesiglas}
%%%%
%%%%CPTEC   &--& Centro de Previsão de Tempo e Estudos Climáticos \\
%%%%INPE    &--& Instituto Nacional de Pesquisas Espaciais \\
%%%%VR      &--& Realidade Virtual \\
%%%%BNCC    &--& Base Nacional Comum Curricular \\
%%%%STEM    &--& Science, Technology, Engineering and Mathematics (Ciência, Tecnologia, Engenharia e Matemática) \\
%%%%MC      &--& Minecraft \\
%%%%VRC     &--& Vivecraft (modificação para VR do Minecraft) \\
%%%%3D      &--& Tridimensional \\
%%%%DIY     &--& Do It Yourself (faça você mesmo) \\
%%%%
%%%%\end{abreviaturasesiglas}
%%%%%%%%%%%%%%%%%%%%%%%%%%%%%%%%%%%%%%%%%%%%%%%%%%%%%%%%%%%%%%%%%%%%%%
%% Autor do prompt: PROFFELIPEELIAS
%% Versão: 1.0
%% Ano de criação: 2025
%%%%%%%%%%%%%%%%%%%%%%%%%%%%%%%%%%%%%%%%%%%%%%%%%%%%%%%%%%%%%%%%%%%%%%


% Abreviaturas e Siglas  %% opcional, mas recomendado

\begin{abreviaturasesiglas}  %% insira abaixo suas abreviaturas conforme o modelo.

%% sigla (separador: &--&) significado (quebra de linha: \\)
CPTEC   &--& Centro de Previsão de Tempo e Estudos Climáticos \\
INPE    &--& Instituto Nacional de Pesquisas Espaciais \\
VR      &--& Realidade Virtual \\
BNCC    &--& Base Nacional Comum Curricular \\
STEM    &--& Science, Technology, Engineering and Mathematics \\
MC      &--& Minecraft \\
VRC     &--& Vivecraft (modificação para VR do Minecraft) \\
3D      &--& Tridimensional \\
DIY     &--& Do It Yourself (faça você mesmo) \\
%
\end{abreviaturasesiglas}
