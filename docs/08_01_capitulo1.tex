%%%%%%%%%%%%%%%%%%%%%%%%%%%%%%%%%%%%%%%%%%%%%%%%%%%%%%%%%%%%%%%%%%%%%%%%%%%%%%%
% Modelo básico de estrutura LaTeX para o capítulo da Introdução com seções, 
% subseções e listas, pronto para uso e com comentários explicativos.
%
% Use este modelo para organizar seu texto dentro do documento .tex
%%%%%%%%%%%%%%%%%%%%%%%%%%%%%%%%%%%%%%%%%%%%%%%%%%%%%%%%%%%%%%%%%%%%%%%%%%%%%%%

%\chapter{INTRODUÇÃO}  % Título do capítulo

% Texto inicial da introdução contextualizando o tema brevemente.
%A divulgação do conhecimento científico para populações jovens enfrenta uma série de desafios complexos. O modelo educacional vigente dificulta o engajamento efetivo dos estudantes~\cite{uol2024}.

% Seção para descrever os objetivos
%\section{Objetivo}

% Sub-seção para o objetivo geral
%\subsection{Objetivo Geral}
%O principal objetivo do projeto CPTECRAFT é democratizar o acesso ao conhecimento científico e tecnológico por meio de um ambiente virtual imersivo dentro do Minecraft.

% Sub-seção para os objetivos específicos, como uma lista
%\subsection{Objetivos Específicos}
%Para alcançar esse objetivo, estabelecemos as metas específicas abaixo:  
%\begin{itemize}  % Início de lista com marcadores
  %\item Construir uma réplica virtual detalhada do CPTEC/INPE no Minecraft;  
  %\item Integrar tecnologia de realidade virtual para ampliar a imersão;  
  %\item Desenvolver óculos VR acessíveis feitos com materiais recicláveis;  
  %\item Realizar oficinas educativas para montagem dos óculos VR e exploração gamificada;  
  %\item Implementar desafios e missões gamificadas para estimular o aprendizado;  
 % \item Avaliar o impacto das atividades por meio de questionários e observação direta;  
%\end{itemize}  % Fim da lista

%%%%%%%%%%%%%%%%%%%%%%%%%%%%%%%%%%%%%%%%%%%%%%%%%%%%%%%%%%%%%%%%%%%%%%%%%%%%%%%
% COMO FAZER CITAÇÕES:
% Use sempre o comando \cite{} para citar referências do seu arquivo .bib, como:
% "... dificuldades enfrentadas~\cite{cardboard2014,araujo2016}."
%
% INSERIR FIGURAS:
% \begin{figure}[H]
%   \centering
%   \includegraphics[width=0.9\textwidth]{caminho/para/imagem.png}
%   \caption{Legenda da figura.}
%   \label{fig:exemplo}
% \end{figure}
%
% INSERIR TABELAS:
% \begin{table}[!ht]
%   \centering
%   \caption{Legenda da tabela}
%   \begin{tabular}{l c c}
%     \hline
%     Coluna 1 & Coluna 2 & Coluna 3 \\
%     \hline
%     Dado 1 & Dado 2 & Dado 3 \\
%     \hline
%   \end{tabular}
%   \FONTE{Fonte dos dados.}
% \end{table}
%%%%%%%%%%%%%%%%%%%%%%%%%%%%%%%%%%%%%%%%%%%%%%%%%%%%%%%%%%%%%%%%%%%%%%%%%%%%%%%

\newpage

\chapter{INTRODUÇÃO}

A divulgação do conhecimento científico para populações jovens enfrenta uma série de desafios complexos e multifacetados que vão desde limitações estruturais do sistema educacional tradicional até obstáculos culturais e sociais. O modelo educacional vigente, caracterizado muitas vezes por uma carga horária restrita e pela predominância do ensino expositivo e mecanicista, dificulta o engajamento efetivo dos estudantes. Além disso, a fragmentação dos conteúdos curriculares e a ênfase na memorização mecânica reduzem as oportunidades para aprendizado crítico e criativo, afastando os jovens do fascínio e do entendimento profundo da ciência~\cite{uol2024}.

Nesse cenário, a popularização dos jogos digitais apresenta uma oportunidade singular de conectar a educação com as tecnologias que fazem parte do cotidiano dos estudantes. Minecraft destaca-se por sua capacidade de estimular habilidades cognitivas essenciais, tais como pensamento estratégico, resolução de problemas e colaboração em equipe, atributos indispensáveis para a formação científica contemporânea e para o desenvolvimento de habilidades do século XXI~\cite{newzoo2024}. A imersão e o ambiente aberto proporcionados pelo jogo permitem que os usuários experimentem processos de planejamento, execução e avaliação de forma lúdica.

Entretanto, é fundamental que essa inserção dos jogos digitais no campo educacional seja feita de forma estruturada e consistente, para que não se torne uma ferramenta meramente recreativa sem impacto pedagógico. A gamificação aplicada com clareza, aliada a objetivos educacionais definidos, pode transformar o aprendizado, potencializando a participação ativa e o interesse genuíno~\cite{santos2022}.

Dessa forma, o projeto CPTECRAFT se posiciona como uma iniciativa que une ciência, tecnologia e inovação, utilizando o Minecraft como plataforma para aproximar jovens das áreas STEM (Ciência, Tecnologia, Engenharia e Matemática). Ao combinar construções virtuais, realidade virtual e oficinas interativas, a proposta busca tornar o conhecimento científico acessível, interessante e inclusivo.

\section{Objetivo}

Os objetivos do projeto são descritos a seguir, divididos em objetivo geral e objetivos específicos.

\subsection{Objetivo Geral}
O principal objetivo do projeto CPTECRAFT é democratizar o acesso ao conhecimento científico e tecnológico por meio do desenvolvimento de um ambiente virtual imersivo dentro do Minecraft, que reproduza o espaço físico do CPTEC/INPE. Essa plataforma procura integrar tecnologias de realidade virtual e metodologias gamificadas, permitindo que crianças e adolescentes, independentemente de sua localização ou condição socioeconômica, explorem conceitos científicos, tecnológicos, de engenharia e matemática de forma interdisciplinar, lúdica e envolvente. Assim, o projeto também tem como meta fomentar a inclusão digital e educacional, preparando os jovens para os desafios científicos e tecnológicos do século XXI.

\subsection{Objetivos Específicos}
Para alcançar esse objetivo, o projeto estabelece as seguintes metas específicas:  
\begin{itemize}
  \item Construir uma réplica virtual cunhada com detalhes arquitetônicos reais do CPTEC/INPE no Minecraft, utilizando a plataforma Forge para customização avançada do ambiente;  
  \item Integrar a tecnologia de realidade virtual via modificação Vivecraft, ampliando a imersão e a interatividade do usuário no espaço virtual;  
  \item Desenvolver modelos acessíveis de óculos VR feitos com materiais recicláveis, visando a redução de custos e a difusão sustentável da tecnologia~\cite{cardboard2014,araujo2016};  
  \item Realizar oficinas educativas com grupos de crianças e adolescentes para que aprendam a montar os óculos VR e explorar o ambiente virtual de forma gamificada;  
  \item Implementar metodologias baseadas em gamificação, com desafios, missões e atividades interativas que estimulem o engajamento, a criatividade e o pensamento científico~\cite{santos2022};  
  \item Avaliar qualitativa e quantitativamente o impacto das oficinas e da plataforma no interesse e no aprendizado dos participantes, utilizando questionários e observação direta;  
\end{itemize}