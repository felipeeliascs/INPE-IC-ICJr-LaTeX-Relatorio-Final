%%%%%%%%%%%%%%%%%%%%%%%%%%%%%%%%%%%%%%%%%%%%%%%%%%%%%%%%%%%%%%%%%%%%%%
%% Sugestão de prompt para IA:
%% "Crie um texto para o ambiente LaTeX abstract (`\begin{abstract} ... \end{abstract}`) em inglês, incluindo o resumo (abstract) e as palavras-chave usando \keywords{...}, com 3 a 5 keywords; mantenha os comandos sem alteração, formate corretamente em LaTeX e evite caracteres especiais sem escape. {COLE O TEXTO OU INFORMAÇÕES AQUI PARA GERAR O ABSTRACT}"
%%%%%%%%%%%%%%%%%%%%%%%%%%%%%%%%%%%%%%%%%%%%%%%%%%%%%%%%%%%%%%%%%%%%%%
%%%%Exemplo:
%%%%\begin{abstract}
%%%%
%%%%Your text here. Your text here. Your text here. Your text here. Your text here. Your text here. Your text here. Your text here. Your text here. Your text here. Your text here. Your text here. Your text here. Your text here. Your text here. Your text here. Your text here. Your text here. Your text here. Your text here. Your text here. Your text here. Your text here. Your text here. Your text here. Your text here. 
%%%%
%%%%\keywords{%
%%%%Virtual Reality,%
%%%%Educational Games,%
%%%%Science,%
%%%%Scientific Outreach%
%%%%}
%%%%\end{abstract}
%%%%%%%%%%%%%%%%%%%%%%%%%%%%%%%%%%%%%%%%%%%%%%%%%%%%%%%%%%%%%%%%%%%%%%
%% Autor do prompt: PROFFELIPEELIAS
%% Versão: 1.0
%% Ano de criação: 2025
%%%%%%%%%%%%%%%%%%%%%%%%%%%%%%%%%%%%%%%%%%%%%%%%%%%%%%%%%%%%%%%%%%%%%%

% ABSTRACT %% mandatory
\newpage
\begin{abstract}
    
%% In this resumo.tex file  
%% the text of the resumo (abstract) and keywords must be in Portuguese for documents written in Portuguese  
%% and must be in English for documents written in English  
%% the command names \begin{resumo}, \end{resumo}, \palavraschave and \palavrachave must not be changed  

%\hypertarget{estilo:resumo}{} %% usage for this Guide

The CPTECRAFT project is an innovative initiative that aims to effectively integrate digital technology and education to bring children and adolescents closer to the world of science. Through an interactive and highly immersive approach, the project proposes the construction of a digital representation of CPTEC/INPE within the Minecraft game, a platform widely recognized for its ability to stimulate creativity and collaboration. We used the Forge platform to enable modifications, such as incorporating virtual reality features. Additionally, we developed an accessible model of virtual reality glasses made from recyclable materials, which not only reduces costs but also encourages sustainable practices among the young participants. In initial tests conducted during the III Young Scientists Meeting at CPTEC/INPE, various age groups participated, making it possible to broadly analyze public reception to the virtual environment. This report presents the results obtained, highlighting the high engagement of participants, the perceptions collected through satisfaction surveys, and future perspectives for expanding the project.

\palavraschave{%
\palavrachave{Virtual Reality}%
\palavrachave{Educational Games}%
\palavrachave{Science}%
\palavrachave{Science Communication}%
}
\end{abstract}

\newpage
