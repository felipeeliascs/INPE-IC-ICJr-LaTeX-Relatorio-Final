%%%%%%%%%%%%%%%%%%%%%%%%%%%%%%%%%%%%%%%%%%%%%%%%%%%%%%%%%%%%%%%%%%%%%%
%% Sugestão de prompt para IA:
%% "Crie um texto para o ambiente LaTeX resumo (`\begin{resumo} ... \end{resumo}`) em português, incluindo o resumo e as palavras-chave usando \palavraschave{ \palavrachave{...} ... }, com 3 a 5 palavras-chave; mantenha os comandos e nomes sem alteração, formate corretamente em LaTeX e evite caracteres especiais sem escape. {COLE O TEXTO OU INFORMAÇÕES AQUI PARA GERAR O RESUMO}"
%%%%%%%%%%%%%%%%%%%%%%%%%%%%%%%%%%%%%%%%%%%%%%%%%%%%%%%%%%%%%%%%%%%%%%
%%%%Exemplo:
%%%%\begin{resumo}
%%%%
%%%%Texto do seu resumo. Texto do seu resumo. Texto do seu resumo. Texto do seu resumo. Texto do seu resumo. Texto do seu resumo. Texto do seu resumo. Texto do seu resumo. Texto do seu resumo. Texto do seu resumo. Texto do seu resumo. Texto do seu resumo.
%%%%
%%%%\palavraschave{%
%%%%\palavrachave{Realidade Virtual}%
%%%%\palavrachave{Jogos Educacionais}%
%%%%\palavrachave{Ciências}%
%%%%\palavrachave{Divulgação Científica}%
%%%%}
%%%%\end{resumo}
%%%%%%%%%%%%%%%%%%%%%%%%%%%%%%%%%%%%%%%%%%%%%%%%%%%%%%%%%%%%%%%%%%%%%%
%% Autor do prompt: PROFFELIPEELIAS
%% Versão: 1.0
%% Ano de criação: 2025
%%%%%%%%%%%%%%%%%%%%%%%%%%%%%%%%%%%%%%%%%%%%%%%%%%%%%%%%%%%%%%%%%%%%%%



% RESUMO %% obrigatório

\begin{resumo}

%% neste arquivo resumo.tex
%% o texto do resumo e as palavras-chave têm que ser em Português para os documentos escritos em Português
%% o texto do resumo e as palavras-chave têm que ser em Inglês para os documentos escritos em Inglês
%% os nomes dos comandos \begin{resumo}, \end{resumo}, \palavraschave e \palavrachave não devem ser alterados

%\hypertarget{estilo:resumo}{} %% uso para este Guia

O projeto CPTECRAFT é uma iniciativa inovadora que visa integrar, de forma efetiva, tecnologia digital e educação para aproximar crianças e adolescentes do universo científico. Por meio de uma abordagem interativa e altamente imersiva, o projeto propõe a construção de uma representação digital do CPTEC/INPE dentro do jogo Minecraft, uma plataforma amplamente reconhecida por sua capacidade de estimular criatividade e colaboração. Utilizamos a plataforma Forge para permitir modificações como incorporar recursos de realidade virtual. Complementarmente, desenvolvemos um modelo acessível de óculos de realidade virtual confeccionados com materiais recicláveis, o que não apenas reduz custos, mas também incentiva práticas sustentáveis entre os jovens participantes. Nos testes iniciais realizados durante o III Encontro de Jovens Cientistas no CPTEC/INPE, diversas faixas etárias participaram, possibilitando uma análise ampla da recepção do público ao ambiente virtual. Este relatório apresenta os resultados obtidos, destacando o elevado engajamento dos participantes, as percepções coletadas por meio de questionários de satisfação e as perspectivas futuras para ampliação do projeto.O projeto CPTECRAFT é uma iniciativa inovadora que visa integrar, de forma efetiva, tecnologia digital e educação para aproximar crianças e adolescentes do universo científico. Por meio de uma abordagem interativa e altamente imersiva, o projeto propõe a construção de uma representação digital do CPTEC/INPE dentro do jogo Minecraft, uma plataforma amplamente reconhecida por sua capacidade de estimular criatividade e colaboração. Utilizamos a plataforma Forge para permitir modificações como incorporar recursos de realidade virtual. Complementarmente, desenvolvemos um modelo acessível de óculos de realidade virtual confeccionados com materiais recicláveis, o que não apenas reduz custos, mas também incentiva práticas sustentáveis entre os jovens participantes. Nos testes iniciais realizados durante o III Encontro de Jovens Cientistas no CPTEC/INPE, diversas faixas etárias participaram, possibilitando uma análise ampla da recepção do público ao ambiente virtual. Este relatório apresenta os resultados obtidos, destacando o elevado engajamento dos participantes, as percepções coletadas por meio de questionários de satisfação e as perspectivas futuras para ampliação do projeto.

\palavraschave{%
\palavrachave{Realidade Virtual}%
\palavrachave{Jogos Educacionais}%
\palavrachave{Ciências}%
\palavrachave{Divulgação Científica}%
}
\end{resumo}